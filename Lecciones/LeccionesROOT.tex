\documentclass[10.5pt]{article}

% Spanish characters
\usepackage[utf8]{inputenc}
\usepackage[T1]{fontenc}
% French display
\usepackage[english,spanish]{babel}

\usepackage{lastpage}
%Esto me permite usar el comando "\pageref{LastPage}" en el footer.
\renewcommand{\baselinestretch}{1.6}
% Esto controla el interlineado o espaciado!!!
\usepackage{color}
%\newcommand{\red}[1]{{\color{red} #1}}
\newcommand{\red}[1]{{\color{black} #1}}

%Esto me permite poner hipervínculos:
%\usepackage[pdftex,
%       colorlinks=true,
%       urlcolor=blue,       % \href{...}{...} external (URL)
%       filecolor=green,     % \href{...} local file
%       linkcolor=black,       % \ref{...} and \pageref{...}
%       pdftitle={Papers by AUTHOR},
%       pdfauthor={Your Name},
%       pdfsubject={Just a test},
%       pdfkeywords={test testing testable},
%%       pagebackref,%Esto parece que pone un numerito al lado de la referencia (en la seccion de bibliografia), donde se puede clicar y te lleva al lugar del texto donde se le cita.
%       pdfpagemode=None,
%       bookmarksopen=true]{hyperref}


%The following packages are relics, but I don't want to remove them just in case:
\usepackage{amsmath}
\usepackage{array}
\usepackage{latexsym}
\usepackage{amsfonts}
\usepackage{amsthm}
\usepackage{cite}
\usepackage{setspace}
\usepackage{amssymb}
\usepackage{hyperref}

\usepackage{multicol}
\usepackage{color}
%\usepackage{minipage}

\usepackage{graphicx} % Required for including images
\graphicspath{{figures/}} % Location of the graphics files
\usepackage[font=small,labelfont=bf]{caption} % Required for specifying captions to tables and figures

%The defaults margins are huge, so I'll customize it:
\oddsidemargin  -0.0 in
\textwidth 6.5 in
\textheight 8.7 in
\addtolength{\voffset}{-1cm}

%%%%%%%%%%%%%%%%%%%%%%%% HEADER AND FOOTER %%%%%%%%%%%%%%%%%%%%
\usepackage{fancyhdr}
\pagestyle{fancy}

\fancyhead[L]{Lecciones ROOT}
%\fancyhead[L]{CNRS Competition 01-04}
\fancyhead[R]{Jos\'{e} David Ruiz \'{A}lvarez}
\fancyhead[C]{}
\fancyfoot[C]{\thepage /\pageref{LastPage}}

\newlength\FHoffset
\setlength\FHoffset{0cm}

\addtolength\headwidth{2\FHoffset}
\fancyheadoffset{\FHoffset}

\newlength\FHleft
\newlength\FHright

\setlength\FHleft{1cm}
\setlength\FHright{1cm}

\thispagestyle{empty}
%%%%%%%%%%%%%%%%%%%%%%%% HEADER AND FOOTER %%%%%%%%%%%%%%%%%%%%



\begin{document}

%\begin{center}
\noindent
\begin{minipage}[b]{0.75\linewidth}
{\LARGE\bf Lecciones ROOT}\\ %[1mm]
%\end{center}
%{\Large\bf \emph{}}\\ %[3mm]
%{\Large\bf \emph{connections between LHC and neutrino experiments}}\\ %[3mm]
%{\Large\bf \emph{from neutrons to Higgses}}\\ %[3mm]
\large{Jos\'{e} David Ruiz \'{A}lvarez} \\
\small{\href{mailto:josed.ruiz@udea.edu.co}{josed.ruiz@udea.edu.co}} \\ %[3mm]
%\normalsize{Plaza código: 2017010307, Área: Física de fenomenología de altas energías} \\%[3mm]
\normalsize{Instituto de Física, Facultad de Ciencias Exactas y Naturales} \\%[3mm]
\normalsize{\bf Universidad de Antioquia} \\[8mm]
\today %\\[4mm]
\end{minipage}%
%\end{center}
%\begin{minipage}[b]{0.25\linewidth}
%\centering{\includegraphics[width=4cm]{figures/CMS.png}}\\
%%%%%\includegraphics[width=15cm]{figures/UniandesColombia.jpg}\\
%\end{minipage}

%\begin{center}
%{\bf Palabras clave:} CERN, LHC, CMS, Materia Oscura
%\end{center}

%\doublespacing

ROOT es un paquete moderno especializado para el análisis de datos que es especialmente utilizado en física de altas energías (física de partículas) y desarrollado en el CERN. Los pasos básicos necesarios para empezar a trabajar con el paquete después de tenerlo instalado están contenidos en los siguientes tutoriales:

\begin{enumerate}
\item \url{https://root.cern.ch/first-steps-root} de donde NO es necesario familiarizarse con PROOF, ni con RooFit ni con TMVA. Tampoco es necesario leer las secciones de ``Simulation in ROOT'' ni ``Parallel processing''.
\item \url{https://root.cern.ch/introductory-tutorials} donde las últimas tres secciones son opcionales.
\end{enumerate}

Adicionalmente, ROOT cuenta con un sistema muy completo donde describe las clases y da ejemplos sobre como utilizarlas. Por ejemplo, para histogramas \url{https://root.cern.ch/doc/master/classTH1.html} y árboles \url{https://root.cern.ch/doc/v612/classTTree.html}.

%\begin{center}
%\includegraphics[width=0.4\linewidth]{DM_detection.jpg}
%\captionof{figure}{Diagrama ilustrativo donde se muestran los 3 procesos principales que se utilizan para la detección de materia oscura. En el sentido de la línea verde se muestra el proceso de producción de materia oscura en colisionadores, la línea roja corresponde a la detección directa y la línea azul a la detección indirecta.}
%\label{fig:DMdetection}
%\end{center}



%\singlespacing
%\begin{thebibliography}{99}
%
%\end{thebibliography}

\end{document}

%%% Local Variables:
%%%   mode: latex
%%%   mode: flyspell
%%%   ispell-local-dictionary: "spanish"
%%% End:
